\documentclass[answers]{exam}

%% Language and font encodings
\usepackage[english]{babel}
\usepackage[utf8x]{inputenc}
\usepackage[T1]{fontenc}

%% Sets page size and margins
\usepackage[a4paper,margin=2cm]{geometry}

%% Useful packages
\usepackage{amsmath}
\usepackage{graphicx}
\usepackage{paralist}
\setlength\FrameSep{4pt}

\begin{document}
{\centering
{\Huge Questionário para Analistas}\\
\vspace{1em}
Rodolfo Labiapari Mansur Guimarães - \texttt{rodolfolabiapari@gmail.com}\\
\textit{Todo o questionário foi respondido sem consulta à nenhum meio de comunicação ou informação, preservando a ética das questões.}

\vspace{0.5cm}
}
\begin{questions}

\question{
Onde ficam, por padrão, as senhas de usuários em sistemas Linux modernos?
}\begin{framed}
As senhas, por padrão, ficam situadas no diretório \texttt{/etc/passwd} e também em \texttt{/etc/shadow}.
\end{framed}



\question{
Como inicializar um daemon de ssh em um sistema Linux (cite como e qual distribuição).
}\begin{framed}
No Ubuntu, para iniciar um daemon de ssh, usa-se o comando \texttt{\$ sshd user@10.0.0.1} em um sistema remoto, por exemplo.
\end{framed}



\question{
Como posso ver a tabela de partição de um disco numa máquina Linux?
}\begin{framed}
Para visutalizar a tabela de partição de um disco, utilizaria-se o comando \texttt{\$ fdisk -l} em linha de comando.
\end{framed}



\question{
O que são bash, ksh, sh, csh?
}\begin{framed}
Estes são interpretadores de linha de comando.

\textit{bash} e \textit{ksh} são derivados do \textit{sh} e \textit{csh} é original do BSD.
\end{framed}



\question{
Como atribuir uma variável de ambiente no shell bash?
}\begin{framed}
Utiliza-se o simbolo = (igual) para atribuição de valores. Como por exemplo:

\texttt{\$ VERSION=4.0.1}
\end{framed}



\question{
Qual a saída do script abaixo?
\begin{verbatim}
$ cd /tmp
$ if [ `ls` ] ;then
$ echo “vazio”
$ else
$ echo “cheio”
$ fi
\end{verbatim}
}\begin{framed}
Este script sempre retornará \texttt{vazio} quando houver 1 ou mais arquivos no diretório e \texttt{cheio} sempre que estiver vazio.

Se executado quando um sistema linux estiver acabado de inicializar, apresentará a palavra \texttt{cheio} pois a pasta de arquivos temporários estará vazia.
\end{framed}



\question{
Como listar as portas abertas e conexões ativas num sistema linux genérico?
}\begin{framed}
\texttt{ }
\end{framed}



\question{
Para que servem os protocolos SMTP, POP3 e IMAP? Defina propriedades de cada um
}\begin{framed}
Os protocolos servem para clientes e servidores realizarem atualizações e envios de e-mails.
\begin{description}
    \item[SMTP:] Protocolo para envio de mensagens de e-mail;
    \item[POP3:] Protocolo para receber as mensagens de e-mail salvando-as no cliente;
    \item[IMAP:] Sincroniza as mensagens e status tanto no servidor no cliente.
\end{description}
\end{framed}



\question{
Como listar as rotas correntes em um sistema linux genérico?
}\begin{framed}
Há duas opções, sendo elas o comando \texttt{\$ route} e também o comando \texttt{\$ netstat}. 
Cada um com seus próprios argumentos específicos para exibição.
\end{framed}



\question{
Em relação a Firewall Iptables qual a função de cada uma das tabelas FILTER, NAT e MANGLE.
}\begin{framed}
As funções são:
\begin{description}
    \item[FILTER:] Regras que são executadas no recebimento do pacote, como aceitação e descarte;
    \item[NAT:] Regras que podem ser entendidas como tradutores de pacotes podendo realizar manuseios como desvios por exemplos;
    \item[MANGLE:] Regras que realizam ações de acordo com o pacote recebido.
\end{description}
\end{framed}



\question{
O que é VPN? Descreva três protocolos ou produtos de VPN utilizados atualmente.
}\begin{framed}
VPN é um protocolo para acesso à redes passando por um meio não seguro.

O SSH é um protocolo que utiliza essa tecnologia.
\end{framed}



\question{
Em virtualização, defina os conceitos de hipervisor e máquina virtual.
}\begin{framed}
Um hipervisor é o sistema ou aplicação na qual gerencia as máquinas virtuais.
Tem o papel de gerenciar os recursos para cada máquina virtual.

Uma máquina virtual é um sistema completo virtualizado na qual utiliza um número limitado de recursos para a execução de serviços específicos.
\end{framed}



\question{
Temos um web server responsável por diversos websites cujo acesso é feito por diversos domínios diferentes,
como por exemplo www.dominio1.com e www.dominio2.com.br. Utilizando  o Webserver Nginx, como seria feita esta configuração? Descreva.
}\begin{framed}
\texttt{ }
\end{framed}



\question{
O que são EC2, IAM, RDS, Route53 e S3 ?
}\begin{framed}
São serviços fornecidos pela Amazon.

\begin{description}
    \item[EC2:] é um serviço de Cloud;
    \item[Route 53:] é um serviço de DNS;
    \item[S3:] é um serviço de armazenamento.
\end{description}
\end{framed}



\question{
Um cliente precisa conectar sua rede local diretamente com a rede de servidores que estão hospedados na Aws. Quais métodos podemos utilizar para fazer esse comunicação entre as redes ?
}\begin{framed}
Pode-se utilizar uma VPN para realizar a comunicação entre as duas redes de forma segura.
A AWS possui um serviço específico para esta situação de conexão.
\end{framed}



\question{
O que é um CDN ? Como funciona um CDN ?
}\begin{framed}
Basicamente, CDN é um servidor que realiza o armazenamento e a re-entrega de conteúdos para as redes próximas à ele.
Suas principais características são reduzir a latência e gargalo de clientes e seus produtos consumidos pela internet.

Um exemplo seria a Amazon permitir o uso de alguns servidores dentro provedores de internet para que estes sejam utilizados como redistribuidores dos serviços Amazon Prime ou da sua loja Amazon.
\end{framed}



\question{
O que é Docker containers ?
}\begin{framed}
Diferentemente da virtualização de máquinas, o container não possui um sistema completo para a sua execução. 
Ao invés disso, ele utiliza o kernel do próprio localhost para a sua execução, tornando um sistema mais enxuto.

Sendo assim, ele engloba somente a aplicação que o desenvolvedor necessita, bem como suas dependências.
\end{framed}



\question{
Descreva a funcionalidade dos seguintes componentes de um cluster Kubernetes
\begin{itemize}
    \item ETCD
    \item API Server
    \item Controller Manager
    \item Scheduler
    \item Kubelet
    \item Kube-Proxy
\end{itemize}
}\begin{framed}
\texttt{ }
\end{framed}



\question{
Explique sucintamente os seguintes objetos em um cluster Kubernetes
\begin{itemize}
    \item Pod
    \item Deployment
    \item Service
    \item Ingress
    \item DaemonSet
    \item Job
\end{itemize}
}\begin{framed}
\texttt{ }
\end{framed}



\question{
Descreva sucintamente o que é rede de pods, rede de serviços e sua relação com deployments e services
}\begin{framed}
\texttt{ }
\end{framed}



\end{questions}

\end{document}