\documentclass{ximera}

\title{Questionário para Analistas}
\author{Rodolfo Labiapari Mansur Guimarães}

\begin{document}
\begin{abstract}
    %This is a Ximera document.
\end{abstract}
\maketitle
\begin{example}
Onde ficam, por padrão, as senhas de usuários em sistemas Linux modernos?
\begin{explanation}
Ficam situadas no diretório \texttt{/etc/passwd} e também em \texttt{/etc/shadow}.
\end{explanation}
\end{example}


\begin{example}
Como inicializar um daemon de ssh em um sistema linux ( cite como e qual
distribuição).
\begin{explanation}
No Ubuntu, eu utilizaria o comando \textt{sshd user@10.0.0.1} para inicializar uma deamon em um sistema remoto por exemplo.
\end{explanation}
\end{example}


\begin{example}
Como posso ver a tabela de partição de um disco numa máquina linux?
\begin{explanation}
Se o sistema possuir um UI, utilizaria o GParted para visualização.
Caso contrário, utilizaria o comando \textt{fdisk -l} para visualizar em linha de comando.
\end{explanation}
\end{example}


\begin{example}
O que são bash, ksh, sh, csh?
\begin{explanation}
São interpretadores de linha de comando.
Lembro ainda que bash e ksh são derivados do sh e csh é original do BSD.
\end{explanation}
\end{example}


\begin{example}
Como atribuir uma variável de ambiente no shell bash?
\begin{explanation}
Utiliza-se o simbolo = para atribuição de valores. Como por exemplo:
\texttt{VERSION=4.0.1}
\end{explanation}
\end{example}


\begin{example}
Qual a saída do script abaixo?
\begin{verbatim}
cd /tmp
if [ `ls` ] ;then
echo “vazio”
else
echo “cheio”
fi
\end{verbatim}
\begin{explanation}
Este script sempre retornará \texttt{vazio} quando houver 1 ou mais arquivos no diretório e \texttt{cheio} sempre que estiver vazio.
\end{explanation}
\end{example}


\begin{example}
Como listar as portas abertas e conexões ativas num sistema linux genérico?
\begin{explanation}
\end{explanation}
\end{example}


\begin{example}
Para que servem os protocolos SMTP, POP3 e IMAP? Defina propriedades de cada um
\begin{explanation}
Servem para clientes e servidores realizarem atualizações e envios de e-mails.
\begin{description}
    \item[SMTP:] Protocolo para envio de mensagens de e-mail;
    \item[POP3:] Protocolo para receber as mensagens de e-mail salvando-as no cliente;
    \item[IMAP:] Não tenho certeza mas sincroniza as mensagens e status tanto no servidor no cliente.
\end{description}
\end{explanation}
\end{example}


\begin{example}
Como listar as rotas correntes em um sistema linux genérico?
\begin{explanation}
Conheço duas opções, sendo elas o comando \texttt{route} e também o comando \texttt{netstat}. Cada um com seus próprios argumentos específicos para exibição.
\end{explanation}
\end{example}


\begin{example}
Em relação a Firewall Iptables qual a função de cada uma das tabelas FILTER, NAT e MANGLE.
\begin{explanation}

\begin{description}
    \item[FILTER:] Regras que são executadas no recebimento do pacote, como aceitação e descarte;
    \item[NAT:] Regras que podem ser entendidas como tradutores de pacotes podendo realizar manuseios como desvios por exemplos;
    \item[MANGLE:] Regras que realizam ações de acordo com o pacote recebido.
\end{description}
\end{explanation}
\end{example}


\begin{example}
O que é VPN? Descreva três protocolos ou produtos de VPN utilizados atualmente.
\begin{explanation}
VPN é um protocolo para acesso à equipamentos e redes passando por um meio não seguro criando um túnel criptografado.
Um exemplo é o protocolo SSH.
\end{explanation}
\end{example}


\begin{example}
Em virtualização, defina os conceitos de hipervisor e máquina virtual.
\begin{explanation}
\end{explanation}
\end{example}


\begin{example}
Temos um web server responsável por diversos websites cujo acesso é feito por diversos domínios diferentes,
como por exemplo www.dominio1.com e www.dominio2.com.br. Utilizando  o Webserver Nginx, como seria feita esta configuração? Descreva.
\begin{explanation}
\end{explanation}
\end{example}


\begin{example}
Oque são EC2, IAM, RDS, Route53 e S3 ?
\begin{explanation}
\end{explanation}
\end{example}


\begin{example}
Um cliente precisa conectar sua rede local diretamente com a rede de servidores que estão hospedados na Aws. Quais métodos podemos utilizar para fazer esse comunicação entre as redes ?
\begin{explanation}
\end{explanation}
\end{example}


\begin{example}
Oque é um CDN ? Como funciona um CDN ?
\begin{explanation}
\end{explanation}
\end{example}


\begin{example}
Oque é Docker containers ?
\begin{explanation}
\end{explanation}
\end{example}


\begin{example}
Descreva a funcionalidade dos seguintes componentes de um cluster Kubernetes
-ETCD
-API Server
-Controller Manager
-Scheduler
-Kubelet
-Kube-Proxy
\begin{explanation}
\end{explanation}
\end{example}


\begin{example}
Explique sucintamente os seguintes objetos em um cluster Kubernetes
-Pod
-Deployment
-Service
-Ingress
-DaemonSet
-Job
\begin{explanation}
\end{explanation}
\end{example}


\begin{example}
Descreva sucintamente o que é rede de pods, rede de serviços e sua relação com deployments e services
\begin{explanation}
\end{explanation}
\end{example}

\begin{problem}
Here is a multiple choice problem.
\begin{multipleChoice}
\choice[correct]{I'm right!}
\choice{I'm not right}
\end{multipleChoice}
\end{problem}

\begin{exercise}
\begin{selectAll}
\choice[correct]{Select me.}
\choice[correct]{Me too!}
\choice{Not me.}
\end{selectAll}
\end{exercise}


\end{document}
